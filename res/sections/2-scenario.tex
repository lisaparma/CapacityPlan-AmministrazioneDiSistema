\newpage

\section{Scenario aziendale} \label{ref:scenario}
	L'instituto ospedaliero Gaetano Pini si presenta come un istituto specializzato nella cura e nella riabilitazione di malattie ortopediche impegnato anche nella didattica e nella ricerca universitaria. \\Ad oggi per svolgere a pieno i propri compiti l'istituto si affida a dei sistemi informatici che sono riassunti in questo paragrafo.
	Sarà così raccolta una fotografia della situazione informatica dell'istituto, riassumendo i dati più significativi e associandoli tra di loro in modo da ottenere tutte le informazioni utili al fine di svolgere un'analisi il più completa possibile.
	
	\subsection{Dati generali}
	La tabella \ref{tab:attivitaMediche} riassume le principali attività mediche svolte dall'istituto spericifando i numeri e suddividendole per categoria. \\
	Da questi dati notiamo che l’istituto è chiaramente focalizzato sulla specializzazione ortopedica-traumatologica e sui successivi servizi di riabilitazione, e i servizi e le tecnologie che si prestano allo svolgimento di attività indispensabili o utili per questi saranno chiaramente considerati maggiormente nell'analisi. \\
	\begin{table}[h] 
		\centering
		\begin{tabular}{|l|l|}
			\hline
			\rowcolor[HTML]{EFEFEF} 
			\textbf{Dati attività mediche} & \\ \hline
			\textbf{N. posti letto totali}					& \textbf{490} \\
			- Ordinari 							& 460 \\
			- Day Hospital 					& 30 	\\ \hline
			\textbf{Ricoveri (mensili)}  				& \textbf{1950}			\\
			- Ortopedia e traumatologia   & 1073 \\
			- Recupero e riabilitazione 	& 134 \\
			- Reumatologia 						& 719 \\
			- Chirurgia vascolare  				& 24 \\ \hline
			\textbf{Prestazioni (mensili)}	  			& \textbf{12.518}		\\
			- Ortopedia e traumatologia   & 2842 \\
			- Recupero e riabilitazione 	& 8137 \\
			- Reumatologia 						& 1426 \\
			- Chirurgia vascolare  				& 113 \\ \hline
			Attività chirurgiche (annuali)  & 14.400		\\ \hline
		\end{tabular}
	\caption{Riassunto dati attività mediche}\label{tab:attivitaMediche}
	\end{table}
	\\ \\
	
	
	
	\subsection{Sistemi applicativi esistenti}
	La seguente tabella descrive i servizi applicativi centralizzati tutt’ora utilizzati per far fronte alle attività dell’istituto. \\
	Ogni riga della tabella contiene un applicativo e per ogni uno viene specificato:
	\begin{itemize}
		\item criticità, indicata con un numero che va da 1 a 5 e definita dall'istituto,;
		\item numero di utenti totali che utilizzano quel determinato applicativo;
		\item numero di utenti che utilizzano quell'applicativo in contemporanea;
		\item servizio per il quale quell'applicativo viene utilizato;
		\item criticità del servizio indicata con un numero che va da 1 a 5.
	\end{itemize}
	\begin{table}[h]
		\begin{tabular}{|c|C{3cm}| C{2cm} |C{1.5cm} |C{2cm} | C{3cm} | C{2cm}|}
			\hline
			\rowcolor[HTML]{EFEFEF} 
			\textbf{N} & \textbf{Sistema applicativo}  & \textbf{Criticità (1-5)} & \textbf{N. utenti} & \textbf{N. utenti attivi}  & \textbf{Servizio} & \textbf{Criticità servizio (1-5)}\\ \hline
			1  & Archivio Clinico		& 4		& 15	& 10 		& Gestione archivio clinico & 4		\\ \hline
			2  & Argos						& 4		& 5	 		& 2 		& Valorizzazione dei DRG & 3		\\ \hline
			3  & Armonia					& 4		& 6		& 3 		& Gestione anatomia patologica & 4		\\ \hline
			4  & Emonet						& 4		& 10	& 2 		& Gestione produzione emoderivanti & 5		\\ \hline
			5  & Aurora Web				& 2		& 650	& 100 		& Gestione pazienti  & 5		\\ \hline
			6  & Gst\_est					& 2		& 650	& 20 		&  & 		\\ \hline
			7  & Gst\_fat					& 2		& 6			& 4 		& Gestione fatturazione & 4		\\ \hline
			8  & Pacs						& 2		& 49		& 20 		& Gestione diagnostica per immagini & 5	\\ \hline
			9  & Elefante					& 2		& 49		& 20 		&  & 		\\ \hline
			10  & Ormawin2000		& 4		& 6			& 3 		& Gestione sale operatorie / materiale protesico & 3		\\ \hline
		\end{tabular}
	\end{table}

	\begin{table}[h]
	\begin{tabular}{|c|C{3cm}| C{2cm} |C{1.5cm} |C{2cm} | C{3cm} | C{2cm}|}
		\hline
		\rowcolor[HTML]{EFEFEF} 
		\textbf{N} & \textbf{Sistema applicativo}  & \textbf{Criticità (1-5)} & \textbf{N. utenti} & \textbf{N. utenti attivi}  & \textbf{Servizio} & \textbf{Criticità servizio (1-5)}\\ \hline
		11  & PowerLab				& 2		& 10		& 5 		& Gestione laboratorio analisi & 5		\\ \hline
		12  & Rap						& 3		& 6			& 4 		& Gestione fatturazione & 4		\\ \hline
		13  & Ican						& 2		& 350 		& 100 		& Piattaforma SISS & 4		\\ \hline
		14  & Aliseo					& 4		& 20 		& 5 		& Gestione delle risorse umane & 4	\\ \hline
		15  & Enco						& 4		& 43 		& 20 		& Gestione amministrazione e magazzini & 4	\\ \hline
		16  & Teseo						& 2		& 10 		& 2 		&  & 	\\ \hline
		17  & Protocollo			& 3		& 15 		& 5 		& Gestione protocollo & 3	\\ \hline
	\end{tabular}
	\caption{Riassunto applicativi esistenti}\label{tab:applicativi}
	\end{table}

	Dalla tabella \ref{tab:applicativi}, in cui sono state associate diverse informazioni per ogni applicativo esistente fornitoci dall'RFP dell'azienda, possiamo ricavare molte informazioni utili per la prossima analisi. \\
	Dall'associazione dei sistemi applicativi utilizzati con il relativo servizio soddisfatto possiamo notare che non sempre il livello di criticità corrisponde. In particolare, nel caso in cui la criticità del sistema applicativo sia inferiore rispetto a quella del servizio soddisfatto, è bene considerare la criticità maggiore per non rischiare di sottostimare applicativi importanti per il business dell'azienda. \\
	Ulteriore attenzione avranno applicativi riguardanti l'area sanitaria, area di particolare criticità non solo perchè rappresenta il servizio principale fornito dalla struttura ma anche perchè risulta importante a livello umanitario rispetto ad applicativi utilizzati nell'area amministrativa. \\
	Alcuni applicativi di criticità abbastanza bassa sembrano non avere un servizio definito a cui rispondono. Nella presente analisi verranno trattati semplicemente come servizi non critici ma per le prossime revisioni è bene discutere con l'istituto del loro reale impiego per essere sicuri che l'analisi svolta sia del tutto conforme alla realtà. \\ \\ \\
	
	

	
	\subsection{Infrastrutture tecnologiche esistenti}
	La tabella sottostante descrive invece i servizi di infrastruttura utilizzati dall’istituto. \\
	Ogni riga della tabella contiene una tecnologia e per ogni una viene specificato:
	\begin{itemize}
		\item criticità, indicata con un numero che va da 1 a 5 e definita dall'istituto;
		\item numero di utenti che utilizzano quella tecnologia in contemporanea;
		\item servizio per il quale viene utilizata;
		\item criticità del servizio indicata con un numero che va da 1 a 5.
	\end{itemize}
	\begin{table}[h!]
		\begin{tabular}{|c|C{3cm}| C{2cm} |C{2.5cm}| C{3cm} | C{2cm}|}
		\hline
		\rowcolor[HTML]{EFEFEF} 
		\textbf{N.} & \textbf{Sistema}  & \textbf{Criticità (1-5)} & \textbf{N. utenti / accessi contemporanei} & \textbf{Servizio} & \textbf{Criticità servizio (1-5)}\\ \hline
		1  & Router internet	& 4		& 200 & Accesso internet & 5		\\ \hline
		2 & NETASQ F1000 & 3 & 200 & Firewall & 5 \\ \hline
		3 & DNS & 2  & 250 & Risoluzione nomi & 5 \\ \hline
		4 & WINS  & 2 & 250 & Risoluzione nomi & 5 \\ \hline
		5 & EXCHANGE & 3  & 150 & Posta elettronica & 5 \\ \hline
		6 & DHCP  & 2 &  250 & Distribuzione indirizzi IP & 5\\ \hline
		7 & NTP  &3 & 250 & Distribuzione tempo & 5\\ \hline
		8 & PORTA APPLICATIVA & 2  & 50 & Accesso SISS & 5 \\ \hline
		9 & ORACLE & 1 &  250 & Condivisione dati & 5 \\ \hline
		10 & BACKUP & 4 &  NA & Gestione backup & 4 \\ \hline
		11 & ACCESSI REMOTO & 3 &  5 & Gestione accessi remoto & 4 \\ \hline
		12 & REPOSITORY & 3 &  100 & Gestione referti & 5 \\ \hline
	\end{tabular}
	\caption{Riassunto tecnologie esistenti}\label{tab:tecnologie}
	\end{table}

Anche nella tabella \ref{tab:tecnologie} notiamo che non sempre il livello di criticità corrisponde tra la tecnologia utilizzata e il servizio che soddisfa e, come è avvenuto per gli applicativi,  nel caso in cui la criticità della tecnologia sia inferiore rispetto a quella del servizio soddisfatto, considereremo più rilevante la criticità maggiore per non sottostimare la reale criticità. \\


Riassumiamo inoltre nella tabella \ref{tab:infrastruttureGenerali} ulteriori tecnologie disponibili idonee soltanto per l’attuale carico di attività:
\begin{table}[h] 
	\centering
	\begin{tabular}{|l|l|}
		\hline
		\rowcolor[HTML]{EFEFEF} 
		\textbf{Infrastruttura} & \\ \hline
		\textbf{N. server totali}					& \textbf{55} \\
		- Criticità 5 				& 12 \\
		- Criticità 4				& 30 \\
		- Criticità 3				& 10 	\\
		- Criticità 2				& 1 	\\ 
		- Criticità 1				& 2 	\\  \hline
		\textbf{N. apparecchiature di rete}  				& \textbf{28}			\\\hline
		\textbf{N. terminali rilevazione presenze}	  			& \textbf{11}		\\\hline
		\textbf{N. terminali controllo accessi}	  			& \textbf{2}		\\\hline
		\textbf{N. Postazioni}					& \textbf{480} \\
		- Sistema Operativo Windows XP Sp2 & \\
		- Antivirus eTrust V. 7.1 & \\
		- SISS v. 9.X.X & \\
		- Sissway & \\
		- MS Office & \\  \hline
	\end{tabular}
	\caption{Riassunto ulteriori tecnologie utilizzate}\label{tab:infrastruttureGenerali}
\end{table}
	