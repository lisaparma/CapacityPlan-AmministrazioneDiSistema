\newpage
\section{Analisi di capacità} \label{ref:capacita}
Come è stato detto l'istituto ospedaliero Gaetano Pini oltre ad essere un istituto specializzato nella cura e nella riabilitazione di malattie ortopediche è anche un eccellenza nella ricerca universitaria. \\
Gli SLA definiti con l'azienda sono molto stringenti. In particolare è bene tenere presente che tutte le stime e le analisi si baseranno su bisogni del campo operativo dell'azienda, quello medico, che è molto delicato. In particolare si terranno presenti:
\begin{itemize}
	\item i requisiti delle norme UNI EN ISO 9001:2000;
	\item gli standards Joint Commission International;
	\item gli obiettivi e alle indicazioni fornite dalla programmazione nazionale, regionale e locale;
	\item per i servizi più critici il servizio dovrà essere disponibile 24/24.
\end{itemize}

\subsection{Previsione dell'utilizzo e delle prestazioni del servizio}
	Per una dettagliata analisi delle risorse IT necessarie è bene che i servizi IT che vengono utilizzati dall'istituto (che sono stati descritti nella sezione \ref{ref:scenario.applicativi}, precisamente nella tabella \ref{tab:applicativi}) siano identificati attraverso il loro utilizzo da parte degli utenti e le loro performace. \\

	\subsubsection{Presupposti}
	Le performace analizzate in questa sezione riguardano l'utilizzo medio dei servizi da parte degli utenti, senza considerare quindi picchi improvvisi e sovraccarichi, incidenti o malfunzionamenti hardware o software. \\
	\subsubsection{Previsioni}
	L'analisi dei servizi tutt'ora proposti insieme alle previsioni di crescita del volume di business ha portato alla definizione di un piano di crescita per i servizi IT.\\
	In dettaglio:
	\begin{itemize}
		\item l'aumento del numero di utenti prevista è dettato da un incremento del personale;
		\item l'aumento degli utenti attivi è dettato da un aumento dei pazienti;
		\item l'aumento delle performance è dovuto al miglioramento dell'infrastruttura tecnologica che ha portato ad un'ottimizzazione, ad un calo degli incidenti e delle interruzioni di servizio.
	\end{itemize}

Inoltre si vuole precisare che le performance per ogni servizio sono calcolate in base a valori dati da:
\begin{itemize}
	\item efficienza del servizio,
	\item efficacia del servizio,
	\item il raggiungimento degli SLA specifici del servizio
\end{itemize}
La soglia minima di performance raggiunta si attesta all’85\%, al di sopra di quanto concordato negli SLA.
	
	Nella tabella \ref{tab:analisiServizi} vengono identificati i servizi IT, il loro utilizzo da
parte degli utenti attuale e quello previsto per il prossimo anno, il numero di utenti e le loro performance attuali e previste.
	\begin{table}[h!]
	\begin{tabular}{|c|C{3cm}|| C{1.5cm} |C{1.5cm} ||C{1.5cm} | C{1.5cm} || C{1.5cm}| C{1.5cm}|}
		\hline
		\rowcolor[HTML]{EFEFEF} 
		\textbf{N} & \textbf{Sistema applicativo}  & \textbf{N. utenti} & \textbf{N. utenti previsti} & \textbf{N. utenti attivi}  & \textbf{N. utenti attivi previsti} & \textbf{Performance di servizio} & Performance di servizio previste \\ \hline
		1  & Archivio Clinico		& 15	& 	& 10	&0  &0	&0	\\ \hline
		2  & Argos			        	&5	&	& 2	&  &	&	\\ \hline
		3  & Armonia					&6	& 	& 3	&  &	&	\\ \hline
		4  & Emonet						& 10	& 	&2	& &		&	\\ \hline
		5  & Aurora Web				& 650	& 	&100	&  &	&		\\ \hline
		6  & Gst\_est					& 650	& 	& 20	&  &	& 		\\ \hline
		7  & Gst\_fat					& 6	&	&4	&  & 	&	\\ \hline
		8  & Pacs						& 49	&	& 20	&	& 	&	\\ \hline
		9  & Elefante					& 49	&	& 20	&  &	& 		\\ \hline
		10  & Ormawin2000		& 6	&	& 3	&  & 	& 		\\ \hline
		11  & PowerLab				& 10	&	& 5	&  &	& 		\\ \hline
		12  & Rap						& 6	&	&	4&  &	& 		\\ \hline
		13  & Ican						& 350	&	&100	&  &	& 		\\ \hline
		14  & Aliseo					& 20	&	&5	&  &	& 		\\ \hline
		15  & Enco					& 43	&	&20	&  &	& 		\\ \hline
		16  & Teseo						& 10	&	&2	&  &	& 		\\ \hline
		17  & Protocollo			& 15	&	& 5	&  &	& 		\\ \hline
	\end{tabular}
\caption{Analisi servizi IT}\label{tab:analisiServizi}
\end{table}
	
\newpage
\subsection{Previsione dell'utilizzo e delle prestazioni delle risorse}
In questa sezione vengono analizate le infrastrutture tecnologiche esistenti e le loro performance che sono stati descritti nella sezione \ref{ref:scenario.tecnologie}, basandoci sulle previsioni fatte. \\
	\subsubsection{Presupposti}
	Le performace analizzate in questa sezione riguardano l'utilizzo medio dei servizi da parte degli utenti (e quindi delle varie tecnologie), senza considerare quindi picchi improvvisi e sovraccarichi, incidenti o malfunzionamenti hardware o software.
\subsubsection{Previsioni}

	\paragraph{Sistemi tecnologici}
		\textcolor{red}{ ... \\  ... \\ ... \\ ... \\  ... \\ ... \\ }
	\begin{table}[h!]
		\begin{tabular}{|c|C{3cm} || C{1.5cm} |C{1.5cm} || C{1.5cm} |C{1.5cm}|| C{1.5cm} |C{1.5cm}|}
			\hline
			\rowcolor[HTML]{EFEFEF} 
			\textbf{N.} & \textbf{Sistema}  & \textbf{CPU attuale} & \textbf{CPU prevista} & \textbf{RAM attuale} & \textbf{RAM prevista} & \textbf{Storage attuale} & \textbf{Storage previsto} \\ \hline
			1  & Router internet	& 	&  &   &  &	 & \\ \hline
			2 & NETASQ F1000 & 	&  &   &  &	 & \\ \hline
			3 & DNS & 	&  &   &  &	 & \\ \hline
			4 & WINS  & 	&  &   &  &	 & \\ \hline
			5 & EXCHANGE & 	&  &   &  &	 & \\ \hline
			6 & DHCP  & 	&  &   &  &	 & \\ \hline
			7 & NTP  & 	&  &   &  &	 & \\ \hline
			8 & PORTA APPLICATIVA & 	&  &   &  &	 & \\ \hline
			9 & ORACLE & 	&  &   &  &	 & \\ \hline
			10 & BACKUP & 	&  &   &  &	 & \\ \hline
			11 & ACCESSI REMOTO & 	&  &   &  &	 & \\ \hline
			12 & REPOSITORY & 	&  &   &  &	 & \\ \hline
		\end{tabular}
		\caption{Analisi tecnologie IT}\label{tab:analisiTecnologie}
	\end{table}


	\paragraph{Server}
	\textcolor{red}{ ... \\  ... \\ ... \\ ... \\  ... \\ ... \\ }
	\begin{table}[h!]
		\begin{tabular}{|C{2cm} || C{1.5cm} |C{1.5cm} || C{1.5cm} |C{1.5cm}|| C{1.5cm} |C{1.5cm}||C{1.5cm} |C{1.5cm}|} 
			\hline
			\rowcolor[HTML]{EFEFEF} 
			\textbf{Server}  & \textbf{CPU attuale} & \textbf{CPU prevista} & \textbf{RAM attuale} & \textbf{RAM prevista} & \textbf{Storage attuale} & \textbf{Storage previsto} & \textbf{Velocità accesso attuale} & \textbf{Velocità accesso previsto}\\ \hline
			Router internet	& 	&  &   &  &	 & & & \\ \hline
		\end{tabular}
		\caption{Analisi server}\label{tab:analisiServer}
	\end{table}


	\paragraph{Servizi di rete}
	\textcolor{red}{ ... \\  ... \\ ... \\ ... \\  ... \\ ... \\ }
		\begin{table}[h!]
		\begin{tabular}{|C{2cm} || C{1.5cm} |C{1.5cm} || C{1.5cm} |C{1.5cm}|| C{1.5cm} |C{1.5cm}||C{1.5cm} |C{1.5cm}|} 
			\hline
			\rowcolor[HTML]{EFEFEF} 
			\textbf{Server}  & \textbf{CPU attuale} & \textbf{CPU prevista} & \textbf{RAM attuale} & \textbf{RAM prevista} & \textbf{Storage attuale} & \textbf{Storage previsto} & \textbf{Velocità accesso attuale} & \textbf{Velocità accesso previsto}\\ \hline
			Router internet	& 	&  &   &  &	 & & & \\ \hline
		\end{tabular}
		\caption{Analisi servizi di rete}\label{tab:analisiRete}
	\end{table}

	\paragraph{Postazioni di lavoro}
	\textcolor{red}{ ... \\  ... \\ ... \\ ... \\  ... \\ ... \\ }
		\begin{table}[h!]
		\begin{tabular}{|C{2cm} || C{1.5cm} |C{1.5cm} || C{1.5cm} |C{1.5cm}|| C{1.5cm} |C{1.5cm}||C{1.5cm} |C{1.5cm}|} 
			\hline
			\rowcolor[HTML]{EFEFEF} 
			\textbf{Server}  & \textbf{CPU attuale} & \textbf{CPU prevista} & \textbf{RAM attuale} & \textbf{RAM prevista} & \textbf{Storage attuale} & \textbf{Storage previsto} & \textbf{Velocità accesso attuale} & \textbf{Velocità accesso previsto}\\ \hline
			Router internet	& 	&  &   &  &	 & & & \\ \hline
		\end{tabular}
		\caption{Analisi postazioni di lavoro}\label{tab:analisiPostazioni}
	\end{table}




