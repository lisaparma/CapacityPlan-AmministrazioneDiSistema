\newpage

\section{Glossario} \label{ref:glossario}

\subsection*{Availability}

Capacità di un servizio IT o di un componente di esplicare le proprie funzioni in un determinato periodo di tempo e sotto certe condizioni.


\subsection*{AppOptics}

Tool di monitoring e trend analysis avanzato utilizzato per le suddette attività; reperibile al link \href{https://www.appoptics.com/}{https://www.appoptics.com/}



\subsection*{Continuity Management}

Insieme di operazioni che guidano l'organizzazione IT a riprendere le sue normali funzioni dopo l'avvenuta sospensione dei servizi causata da un incidente o problema.


\subsection*{Data Center}
Centro elaborazione dati, sicuro secondo vari livelli chiamati Tier, in cui sono contenuti vari server al servizio di più aziende.


\subsection*{Hardware}
Insieme delle componenti fisiche, non modificabili, di un sistema di elaborazione dati.

\subsection*{Incidente} 
qualcosa che succede anche se non dovrebbe succedere (inaspettato), ogni evento che non rientra nel servizio concordato


\subsection*{Maintainability}

Capacità di un componente o servizio IT di essere ripristinato.
Definita come la probabilità di un componente di essere riparato con successo dato uno specifico intervallo di tempo.

\subsection*{Monitoring}
Attività di monitoraggio di un componente o di un'erogazione di servizio IT con un tool specifico che permetta di raccogliere dati e produrre grafici secondo le informazioni ottenute dal monitoraggio.

\subsection*{OLA}

\textbf{Operational Level Agreement}: accordo interno che copre la fornitura di servizi, che supporta l’IT directorate nella loro erogazione dei servizi.


\subsection*{Reliability}
Capacità di un componente o servizio IT di continuare a svolgere il suo compito sotto determinate condizioni.


\subsection*{RFP}
\textbf{Request For Proposal}: documento che spiega il lavoro che viene richiesto ad un'azienda interessata ad ottenerne l'appalto. In questo documento l’azienda comunica la disponibilità di risorse disponibili per un particolare progetto o programma ed i fornitori possono presentare offerte per il completamento del progetto.

\subsection*{Risk Analysis CRAMM}
Metodologia di analisi dei rischi che si compone essenzialmente in tre parti:
\begin{itemize}
    \item identificazione degli asset di valore;
    \item valutazione del rischio per ogni servizio/componente IT;
    \item identificazione delle contromisure da adottare.
\end{itemize}

\subsection*{Server}
Componente o sottosistema informatico di elaborazione e gestione del traffico di informazioni che fornisce, a livello logico e fisico, un qualunque tipo di servizio ad altre componenti.

\subsection*{SLA}

\textbf{Service Level Agreement}: contratto tra cliente e fornitore che definisce le metriche del servizio che il fornitore deve rispettare.

\subsection*{Sistema informatico}
Infrastruttura tecnologica sulla quale poggia il sistema informativo.


\subsection*{SISS}

\textbf{Sistema Informativo Socio Sanitario}: piattaforma che permette agli operatori sanitari di accedere ai servizi dedicati.




\subsection*{Trend Analysis}

Tecnica di predizione del comportamento di strutture IT basato sui dati raccolti dal monitoraggio.


\subsection*{UnAvailability}
Assenza del servizio dovuta ad incidenti/problemi.

\subsection*{Underpinning Contracts}
Contratto tra il fornitore di servizi e un'azienda esterna. L'azienda esterna offre vari servizi che supportano il fornitore nella sua erogazione del servizio al cliente. 


\subsection*{VBF}

\textbf{Vital Business Function}: sono le funzioni del reparto IT indispensabili e critiche per lo svolgimento del servizio offerto dal business.

