\newpage
\section{Analisi di capacità}

	\textcolor{red}{In questa sezione verrà analizzato lo scenario descritto nella sezione 2 in modo da poter determinare i requisiti di capacità.}.


\subsection{Previsione dell'utilizzo e delle prestazioni del servizio}
	\textcolor{red}{	Tutti i dettagli su tutti i livello di servizio monitorati.}
	
	\subsubsection{Presupposti}
	\textcolor{red}{Ipotesi su cui si basa la previsione }
	\subsubsection{Previsioni}
	\textcolor{red}{}
	\subsubsection{Misure per far fronte alla domanda aggiuntiva}
	
	
\subsection{Previsione dell'utilizzo e delle prestazioni delle risorse}


\newpage
\section{Analisi di impatto}
	\textcolor{red}{Ci sono fattori che possono avere un impatto sulla gestione della capacità e quindi sulla formilazione del capacity plan? }
	\subsection{Nuovi applicazioni}
	\textcolor{red}{Durante l'anno si vogliono o si devono cambiare applicativi?}
	\subsection{Nuove tecnologie}
	\textcolor{red}{Durante l'anno si vogliono o si devono cambiare tecnologie?}
	\subsection{Variazioni contrattuali}
	\textcolor{red}{Dati i molti contratti in essere?}
	
\newpage
\section{Analisi del rischio}
\textcolor{red}{Analisi del rischio tramite il metodo CRAMM relativa alla capacità dei servizi forniti}

\newpage
\section{Costi}
\textcolor{red}{Fornire una tabella che elenchi in modo approssimativo i costi che bisognerebbe sostenere per i cambiamenti richiesti}
