\newpage
\section{Analisi di capacità} \label{ref:capacita}
Come è stato detto l'istituto ospedaliero Gaetano Pini oltre ad essere un istituto specializzato nella cura e nella riabilitazione di malattie ortopediche è anche un eccellenza nella ricerca universitaria. \\
Gli SLA definiti con l'azienda sono molto stringenti. In particolare è bene tenere presente che tutte le stime e le analisi si baseranno su bisogni del campo operativo dell'azienda, quello medico, che è molto delicato. In particolare:
\begin{itemize}
	\item 
\end{itemize}

Considerando i trend di crescita rilevati durante la fase di modelling e Business Management l'attenzione viene posta su...
\\
\\
Le previsioni sui servizio definiscono quindi la necessita di nuove componenti hardware e software...
\\

\subsection{Previsione dell'utilizzo e delle prestazioni del servizio}
	Per una dettagliata analisi delle risorse IT necessarie è bene che i servizi IT che vengono utilizzati dall'istituto (che sono stati descritti nella sezione \ref{ref:scenario.applicativi}, precisamente nella tabella \ref{tab:applicativi}) siano identificati attraverso il loro utilizzo da parte degli utenti e le loro performace. \\
	\\
	...In particolare vediamo la scalabilità, la velocità effettiva, i requisiti di disponibilità, l'archiviazione, la sicurezza, i backup.
	Delineamo la strategia di sviluppo di questi scenari.
	
	Descrivere la capacità attualmente disponibile, descrivere le aspettative di crescita future e pianificate.

	\subsubsection{Presupposti}
	Le performace analizzate in questa sezione riguardano l'utilizzo medio dei servizi da parte degli utenti, senza considerare quindi picchi improvvisi e sovraccarichi, incidenti o malfunzionamenti hardware o software.
	\subsubsection{Previsioni}
	
	\begin{table}[h!]
		\begin{tabular}{|c|C{3cm}|| C{1.5cm} |C{1.5cm} ||C{1.5cm} | C{1.5cm} || C{1.5cm}| C{1.5cm}|}
			\hline
			\rowcolor[HTML]{EFEFEF} 
			\textbf{N} & \textbf{Sistema applicativo}  & \textbf{N. utenti} & \textbf{N. utenti previsti} & \textbf{N. utenti attivi}  & \textbf{N. utenti attivi previsti} & \textbf{Performance di servizio} & Performance di servizio previste \\ \hline
			1  & Archivio Clinico		& 0	& 0	&0	&0  &0	&0	\\ \hline
			2  & Argos			        	&	&	&	&  &	&	\\ \hline
			3  & Armonia					&	& 	&	&  &	&	\\ \hline
			4  & Emonet						& 	& 	&	& &		&	\\ \hline
			5  & Aurora Web				& 	& 	&	&  &	&		\\ \hline
			6  & Gst\_est					& 	& 	&	&  &	& 		\\ \hline
			7  & Gst\_fat					& 	&	&	&  & 	&	\\ \hline
			8  & Pacs						& 	&	&	&	& 	&	\\ \hline
			9  & Elefante					&	&	&	&  &	& 		\\ \hline
			10  & Ormawin2000		& 	&	&	&  &	& 		\\ \hline
			11  & PowerLab				& 	&	&	&  &	& 		\\ \hline
			12  & Rap						& 	&	&	&  &	& 		\\ \hline
			13  & Ican						& 	&	&	&  &	& 		\\ \hline
			14  & Aliseo					& 	&	&	&  &	& 		\\ \hline
			15  & Enco					& 	&	&	&  &	& 		\\ \hline
			16  & Teseo						& 	&	&	&  &	& 		\\ \hline
			17  & Protocollo			& 	&	&	&  &	& 		\\ \hline
		\end{tabular}
	\end{table}
	
	\subsubsection{Misure per far fronte alla domanda aggiuntiva}
	
\newpage
\subsection{Previsione dell'utilizzo e delle prestazioni delle risorse}

	\subsubsection{Presupposti}
Le performace analizzate in questa sezione riguardano l'utilizzo medio dei servizi da parte degli utenti, senza considerare quindi picchi improvvisi e sovraccarichi, incidenti o malfunzionamenti hardware o software.
\subsubsection{Previsioni}

	\paragraph{Sistemi tecnologici}
	\paragraph{Server}
	\paragraph{Servizi di rete}
	\paragraph{Postazioni di lavoro}




