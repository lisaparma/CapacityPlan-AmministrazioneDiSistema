\newpage

\section{Costi} \label{ref:costi}
Riassumiamo di seguito un'approssimazione dei costi che l'azienda dovrebbe effettuare per portare a compimento il piano redatto. \\

I costi nella tabella \ref{tab:costiHardware} riguardanti le tecnologie da acquistare sono indicativi, basati sul prezzo di mercato ad oggi.\\
	\begin{table}[h!]
		\centering
	\begin{tabular}{|C{3cm} || C{2cm} | C{2cm}| C{2cm} |} 
		\hline
		\rowcolor[HTML]{EFEFEF} 
		\textbf{Componente}  &\textbf{Quantità attuale} &\textbf{Quantità prevista} & \textbf{Costo} \\ \hline
		Server & 10 & 14 & 4 x 500 euro\\ \hline 
		CPU & 2 & 3 & 1 x 800 euro \\ \hline 
		RAM media & 16GB & 32GB & 200 euro x4 \\ \hline 
		Storage & 2 & 3 & 70 euro x 15 \\ \hline 
	\end{tabular}
	\caption{Analisi costi}\label{tab:costiHardware}
\end{table} \\
Per un costo totale di circa 4.650 euro per le nuove tecnologie da acquistare per sostenere il nuovo volume di business. \\
In aggiunta è da calcolare il costo per la manutenzione di queste nuove tecnologie e di quelle già esistenti, descritte nella tabella \ref{tab:tecnologie}.