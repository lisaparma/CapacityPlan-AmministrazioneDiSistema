\newpage

\section{Introduzione} \label{intro}
	Con il presente documento si vuole fornire una stima delle risorse IT necessarie per supportare le funzionalità e i requisiti di prestazione delle varie applicazioni utilizzate e servizi forniti dall'istituto  Ospedaliero Gaetano Pini.\\
	Il piano di capacità tiene conto dei servizi e delle tecnologie utilizzate ad oggi e delle previsioni della domanda dell'istituto, stimando costi per il raggiungimento degli obbiettivi concordati in termini di livello di servizio. \\
	Le stime effettuate sono frutto dell'osservazione dell'attuale sistema con l'obbiettivo di un approvvigionamento ottimale ed economico delle risorse e dei servizi allineandoli alle esigenze aziendali.
	
	\subsection{Struttura documento}
	Il seguente documento è strutturato come segue:
	\begin{itemize}
		\item La sezione \ref{ref:scenario}, \nameref{ref:scenario}, elenca la struttura ad oggi dei servizi e delle infrastrutture tecnologiche utilizzate, ne descrive il campo applicativo, l'utilizzo e la criticità.
		\item La sezione \ref{ref:capacita}, \nameref{ref:capacita}, analizza le informazioni della precedente sezione e i piani di crescita in modo da poter determinare i requisiti di capacità, considerando scalabilità, velocità, disponibilità e sicurezza.
		\item  \ref{ref:impatto}, \nameref{ref:impatto}
		\item \ref{ref:rischio}, \nameref{ref:rischio}
		\item \ref{ref:costi}, \nameref{ref:costi}
	\end{itemize}
	\subsection{Assunzioni}
	\textcolor{red}{Eventuali ipotesi / assunzioni fatte durante l'analisi vanno specificate qui}